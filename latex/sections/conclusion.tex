% ============================================================
% CONCLUSION.TEX - Kết luận
% ============================================================

\section{Kết luận}
\label{sec:conclusion}

\subsection{Tóm tắt kết quả}
\label{subsec:summary}

Chúng tôi đã triển khai thành công hệ thống phân tích Petri Net cho 1-safe nets:

\begin{enumerate}
    \item \textbf{Task 1 - PNML Parser}: Sử dụng thư viện \textbf{TinyXML2} để đọc file PNML theo chuẩn quốc tế, xây dựng cấu trúc \code{Model} với ma trận \code{Pre}, \code{Post} và marking khởi đầu $\marking{M}_0$.
    
    \item \textbf{Task 2 - Explicit Reachability}: Triển khai BFS (queue) và DFS (stack) với \code{unordered\_set<Marking>} dùng custom \code{MarkingHash}. Hàm \code{isEnabled()} và \code{fire()} implement đúng định nghĩa toán học.
    
    \item \textbf{Task 3 - Symbolic Reachability}: Sử dụng thư viện \textbf{BuDDy} để encode markings và transition relation. Thuật toán fixpoint với \code{bdd\_relprod()} (image computation) và \code{bdd\_replace()} (rename variables).
    
    \item \textbf{Task 4 - Deadlock Detection}: Mô hình ILP với \textbf{GLPK} tìm marking không có transition enabled. Ràng buộc: $\sum \pre[p][t] \cdot M[p] \leq \sum \pre[p][t] - 1$. Kết hợp BDD để verify reachability.
    
    \item \textbf{Task 5 - Optimization}: Maximize $\mathbf{c}^T \cdot \marking{M}$ với cutting-plane method, loại trừ candidates không reachable.
\end{enumerate}

Hệ thống được test trên 15 mô hình từ đơn giản đến phức tạp (Dining Philosophers, Producer-Consumer, Readers-Writers).

\subsection{Khó khăn và giải pháp}
\label{subsec:challenges}

\begin{enumerate}
    \item \textbf{Tích hợp BuDDy}: 
    \begin{itemize}
        \item Vấn đề: BuDDy có cả C API (\code{BDD} type) và C++ wrapper (\code{bdd} class)
        \item Giải pháp: Lưu BDD root ID vào \code{void* internalState}, dùng \code{bdd\_addref/delref} quản lý memory
    \end{itemize}
    
    \item \textbf{Cross-platform}: 
    \begin{itemize}
        \item Vấn đề: Memory measurement chỉ hoạt động trên Linux (\code{/proc/self/statm})
        \item Giải pháp: Dùng \code{\#ifdef \_WIN32} cho conditional compilation
    \end{itemize}
    
    \item \textbf{GLPK optional}: 
    \begin{itemize}
        \item Vấn đề: GLPK không có sẵn trên mọi hệ thống
        \item Giải pháp: CMake detect GLPK, định nghĩa macro \code{HAS\_GLPK} để conditional compile
    \end{itemize}
\end{enumerate}

\subsection{Hướng phát triển}
\label{subsec:future-work}

\begin{itemize}
    \item \textbf{Dynamic Variable Reordering}: Tích hợp \code{bdd\_reorder()} để tối ưu BDD size
    \item \textbf{Parallel Reachability}: Song song hóa BFS/DFS với multi-threading
    \item \textbf{State Equation Constraints}: Thêm vào ILP để tăng pruning power
    \item \textbf{Symbolic Model Checking}: Mở rộng kiểm tra liveness, safety properties
\end{itemize}

\subsection{Kết luận chung}
\label{subsec:final}

Bài tập lớn giúp nhóm hiểu sâu về:
\begin{itemize}
    \item Lý thuyết Petri Net và ứng dụng trong mô hình hóa hệ thống đồng thời
    \item Phương pháp explicit (BFS/DFS) và symbolic (BDD) cho reachability analysis
    \item Kỹ thuật kết hợp ILP với BDD để giải bài toán tối ưu trên không gian trạng thái
    \item Thực hành triển khai với thư viện TinyXML2, BuDDy, GLPK
\end{itemize}

Hệ thống có tính modular cao, dễ mở rộng, đáp ứng đầy đủ yêu cầu bài tập lớn.
